\chapter{Introduction}
\todo[inline]{The introduction was written pretty long ago, maybe rewrite it to be less
specifically tied to the German dataset}
The Political Mashup
project\footnote{http://search.politicalmashup.nl/about.html} aims to digitize
the world's political proceedings in order to make them easily accessible and
searchable. Unfortunately, the published documents are often primarily intended
to be human-readable, without the embedded semantic structure required to
properly index this data in a digital way. This semantic information is
currently recovered using rule-based methods. Since the data gets transcribed by
a human typist, compiled to a PDF, and then goes back into an imperfect PDF
decompiler, there is a lot of room for minor variations in the output even
though the layout of the document itself is consistent. Dealing with this in a
rule-based system entails using either broad rules that lead to a larger
probability of false positives, or a large amount of narrow rules which can
quickly lead to a spaghetti-like mess of special cases and is very fragile to
unseen issues.

I propose that by using a small number of manually annotated documents as a
dataset, a machine learning algorithm can learn to classify sentences in a way
that allows it to segment a document into its constituent parts, while being
more robust to noise than its rule-based counterpart. The common ways to do
sentence classification (e.g. convolutional neural networks \citep{kim2014conv}
, recurrent neural networks or the simpler bag-of-words models) operate on
sentences in a vacuum, considering only their linguistic contents and ignoring
any contextual information that might be present. This is to be expected
considering that most of the common datasets in this area really \emph{are} just
small bits of text in a vacuum --- often-used datasets involve Twitter messages
or short product reviews. In this case however, the sentences come from a
document with a rich structure providing a lot of context. Anecdotally, as a
human it is trivial to discern section headers in a document even when the
document is in a foreign language; simply the fact that the section header might
be printed in bold and centered rather than left-aligned gives it away.
Incorporating this structural data into the learning propose will hopefully
increase the performance of the system, either by simply scoring better on the
used metrics, or perhaps more indirectly by requiring less data or training time
to achieve the same score.

%%% Local Variables:
%%% mode: latex
%%% TeX-master: "report"
%%% End:
