\section{Related Work}
\todo[inline]{Todo: Expand section}

Various forms of convolutional neural networks are commonly used for text
classification. The most basic architecture is described by
\textcite{kim2014conv}, where the input words are tokenized and embedded before
passing them to the convolutional neural network. Additional exploration of the
parameter space and its effect on various datasets is done by
\textcite{zhang2015conv}. Comparable results are achieved by
\textcite{zhang2015character} by operating on the character-level rather than
the word-level, bypassing the issues overhead of using word embedding (either in
extra training time or in finding suitable pretrained embeddings). All the
previously mentioned architectures use a single convolutional layer; this is
somewhat contrary to current trends in computer vision, where popular models
such as ResNet\citep{resnet2015} go as deep as 152 layers. This difference is
explored by \textcite{Conneau2016ldeep}, who take a character-level CNN and show
that adding more layers improves performance, before leveling out at 29 layers.
They hypothesize that the difference in effective depth between computer vision
and language processing might be due to the difference in datasets. The common
ImageNet dataset used in computer vision deals with 1000 classes; in contrast,
sentiment analysis datasets vary between 2 and 25 classes. In addition, they
note that the deeper networks do require a larger amount of data to train.

In terms of analyzing document structure, \textcite{klampfl2014unsupervised}
introduce a method to analyze scientific articles, detecting blocks of text,
labeling them (as e.g. section headers, tables or references) and determining
the reading order --- all in an unsupervised manner. While their approach to
block detection forms an integral part of this thesis, the rest is too
specifically tied to the format of scientific articles to be applicable in this
scenario.

%%% Local Variables:
%%% mode: latex
%%% TeX-master: "report"
%%% End: