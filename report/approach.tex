\section*{Approach}
%\FloatBarrier

The system consists of two separate parts; an unsupervised algorithm for
augmenting data, and a supervised algorithm for classifying the data (Figure
\ref{fig:overview}). The unsupervised portion attempts to augment the data with
additional structural information. It could be considered a preprocessing step,
with the choice of parameters acting as a way to inject some amount of domain
knowledge into the data. The supervised portion consists of a regular
classification algorithm.

\begin{figure}[htbp]
  \centering
  \missingfigure[figwidth=\textwidth]{Layout of the system}
  \caption{A high-level overview of the system}
  \label{fig:overview}
\end{figure}

\subsection*{Unsupervised}
The unsupervised algorithm attempts to detect blocks of text in the PDF file,
and then clusters those blocks by their similarity (see Figure
\ref{fig:clustered} for an example). This approach is based on work by
\textcite{klampfl2014unsupervised}.
\todo{Explain hierarchical agglomerative clustering and why single-linkage is
  being used}

\begin{figure}[htbp]
  \centering
  \missingfigure[figwidth=\textwidth]{Screenshot of clustered pdf}
  \caption{An example of clustered blocks of text, blocks with the same outline
    color belonging to the same cluster.}
  \label{fig:clustered}
\end{figure}

\subsection*{Supervised}