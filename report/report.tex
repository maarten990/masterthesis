\documentclass[a4paper,12pt]{article} 

\usepackage[T1]{fontenc}
\usepackage{geometry}
\usepackage[english]{babel}
\usepackage{amsmath}
\usepackage{graphicx}
\usepackage{subcaption}
\usepackage{placeins}
\usepackage[hidelinks]{hyperref}
\usepackage{microtype}
\usepackage{import}
\usepackage{todonotes}
\usepackage{pgf}
\usepackage{listings}
\usepackage{color}
\usepackage{forest}
\usepackage{lineno}
\usepackage{booktabs}
\usepackage{multirow}
\usepackage{siunitx}
\usepackage{cleveref}
\usepackage{chngpage}
\usepackage{floatpag}
\usepackage[
    backend=biber,
    natbib=true,
    style=nature,
]{biblatex}
\usepackage[
    margin=10pt,
    font=small,
]{caption}

\definecolor{mygreen}{rgb}{0,0.6,0}
\definecolor{mygray}{rgb}{0.5,0.5,0.5}
\definecolor{mymauve}{rgb}{0.58,0,0.82}
\lstset{
  backgroundcolor=\color{white},   % choose the background color
  basicstyle=\footnotesize,        % size of fonts used for the code
  breaklines=true,                 % automatic line breaking only at whitespace
  captionpos=b,                    % sets the caption-position to bottom
  commentstyle=\color{mygreen},    % comment style
  escapeinside={\%*}{*)},          % if you want to add LaTeX within your code
  keywordstyle=\color{blue},       % keyword style
  stringstyle=\color{mymauve},     % string literal style
}

\newtheorem{nullhypothesis}{Null Hypothesis}
\newtheorem{researchquestion}{Research Question}

\addbibresource{library.bib} 

\begin{document}
\floatpagestyle{plain}
\newcommand{\vect}[1]{\ensuremath{\mathbf{#1}}}
\newcommand{\loss}[1]{\ensuremath{\nabla_{\theta_{#1}} \mathcal{L}(\theta_{#1})}}

\import{.}{title-page-ai.tex}
\tableofcontents
\clearpage
\linenumbers

\abstract{Proceedings of parliamentary debates are often published as
  unstructured PDF files, making them unsuitable for indexing into a database or
  querying for specific information. Digitizing them into a structured format is
  challenging; doing so manually is labor-intensive, doing so through a
  rule-based system is error-prone. Manually annotating a small number of
  documents can provide a training set that allows a convolutional neural
  network to accurately classify the elements that denote the structure of the
  document (e.g.\ the start of a new speech), using purely the textual content
  of the document. Adding a preprocessing step that clusters pieces of text based on
  the physical layout of the document improves the classification performance in
  a minor, but statistically significant way.}

\import{.}{introduction.tex}
\import{.}{problemstatement.tex}
\import{.}{relatedworks.tex}
\import{.}{approach.tex}
%\import{.}{setup.tex}
\import{.}{results.tex}
\import{.}{conclusion.tex}

\printbibliography
\end{document}
